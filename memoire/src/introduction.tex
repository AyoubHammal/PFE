\chapter{Introduction générale}
L'Internet des objets est un sujet émergent d'importance technique, sociale et économique. Les produits de consommation, les biens durables, les voitures et les camions, les composants industriels et utilitaires, les capteurs et autres objets du quotidien sont combinés à la connectivité Internet et à de puissantes capacités d'analyse de données qui promettent de transformer la façon dont nous travaillons, vivons et jouons. Les projections de l'impact de l'IoT sur Internet et l'économie sont impressionnantes, certains prévoyant jusqu'à 100 milliards d'appareils connectés IoT et un impact économique mondial de plus de 11 billions de dollars d'ici 2025.\par
Nous proposons dans ce travail une nouvelle technique d'allocation de ressources dans un environnement Fog/Edge Computing visant à servir de manière optimale les demandes de services générées par un ensemble d'objets IoT. Nous exploitons dans cette technique l'algorithme de correspondance de Gale-Shapley prouvé stable et efficace.\par
Nous commençons d'abord par définir les concepts clés de la problématique et établir l'état de l'art concernant ce domaine de recherche. Puis dans un second chapitre, nous présentons notre technique, le scénario d'exécution et les différents échanges entre les entités de notre environnement. Enfin, nous proposons dans le dernier chapitre une implémentation de cette technique dans un environnement de simulation, ainsi qu'une discussion et une analyse des résultats obtenus.