\chapter{Introduction générale}
\thispagestyle{fancy}
\par L'Internet des objets est un sujet émergent d'importance technique, sociale et économique. La perspective d’une vie quotidienne plus confortable, d’une économie et d’une administration plus efficaces, d'une circulation routière plus sûre, d’un approvisionnement énergétique plus respectueux de l'environnement et d’une vie plus saine favorise son développement. Les produits de consommation, les biens durables, les voitures, les composants industriels et utilitaires, les capteurs et autres objets du quotidien sont combinés à la connectivité Internet et à de puissantes capacités d'analyse de données qui promettent de faciliter tous les aspects de notre vie. Les projections de l'impact de l'\emph{IoT} sur Internet et l'économie sont impressionnantes, certains prévoyant jusqu'à 100 milliards d'objets connectés IoT et un impact économique mondial de plus de 11 milliards de dollars d'ici 2025. \par
La croissance fulgurante du nombre d'objets connectés ainsi que la quantité de données générées et par ces derniers ont mis évidence certains problèmes liés au Cloud Computing, et particulièrement liés à son architecture centralisée. Nous pouvons citer par exemple le problème de surcharge, ou encore la dégradation des performances réseau.\par
Pour pallier à ces problèmes, une alternative au paradigme Cloud a été proposée en 2012 se présentant comme étant une extension de ce dernier nommé "Fog Computing". Cette nouvelle approche permet de décharger le Cloud d'une partie des services qu'il propose en utilisant les capacités limitées dont disposent les ressources intermédiaires entre les objets connectés et le Cloud, d'où le terme Fog.\par
Pour cette raison, le Fog Computing semble être un moyen prometteur pour améliorer les performances réseau, limiter les points de surcharges et réduire la latence, ce qui permet de répondre aux exigences de certaines applications sensibles à l'état du réseau, notamment les applications à temps réel.\par
Cependant, le passage du Cloud Computing au Fog Computing n’est pas si simple. Les ressources intermédiaires ayant des capacités limitées, il est important de disposer d'un mécanisme de gestion de ressource permettant de gérer efficacement les ressources disponibles.
\par Dans ce travail, nous proposons une nouvelle technique d'allocation de ressources dans un environnement \emph{Fog/Edge Computing} visant à servir de manière optimale les demandes de services générées par un ensemble d'objets \emph{IoT}. Nous exploitons dans cette technique l'algorithme de correspondance de Gale-Shapley prouvé stable et efficace.\par
Nous commencerons d'abord par définir les concepts clés de la problématique et établir l'état de l'art concernant ce domaine de recherche. Puis dans un second chapitre, nous présenterons notre technique, le scénario d'exécution et les différents échanges entre les entités de notre environnement. Enfin, nous proposerons dans le dernier chapitre une implémentation de cette technique dans un environnement de simulation, ainsi qu'une discussion et une analyse des résultats obtenus.
