\chapter{Introduction}

\section{Internet of Things}

\subsection{Définition}
Le terme Internet des Objets décrit le réseau d'appareils physiques - souvent hétérogène - interconnectés, et dont le rôle principal est la récolte et l'échange d'information, ainsi que l'interaction avec l'environnement extérieur [1]. Cette infrastructure est dite intelligente, dotée de la capacité de s'auto-organiser, partager l'information de manière optimale et de réagir aux changements environnementaux [2].
Les objets dans ce type de réseau sont de capacité limitée, que ce soit en puissance de calcul ou en consommation d'énergie. Ce sont majoritairement des objets électroniques quotidiens (à l'exemple de smartphones, véhicules ou équipements ménagers), chacun avec sa propre identité, pour communiquer avec le reste des objets, et synchroniser les efforts de réponses aux différentes situations externes auxquelles ils sont exposés.

\subsection{Histoire et Évolution}
L'idée d'interconnecter des dispositifs électroniques moyennant un réseau informatique est apparue dans les années 80s, à l'Université de Carnegie Melon, où on avait relié un distributeur de boissons fraîches à un ordinateur de monitorage. Ce qui a permis aux programmeurs de consulter la disponibilité des boissons à distance et d'éviter les trajets inutiles [2].
Cependant, le premier à avoir introduit ce concept est Kevin Ashton en 1999, qui travaillait dans l'optimisation de chaîne de production chez Procter \& Gamble. Dans sa présentation portant le nom de la technologie, il a développé l'idée de décharger l'homme de la tâche de récolte d'information (en 1999, 50 petabytes d'information était créée par des êtres humains), et de plutôt exploiter la masse d'engins et capteurs disponibles déjà déployés à ce moment là [3].
Cette infrastructure n'a été adoptée en industrie qu'en fin années 2000s, où le ratio d'objets/hommes est passé à 1.84 pour 0.08 en 2003. Depuis, les constructeurs du domaine de téléinformatique concentrent leurs efforts sur la production de capteurs et d'appareils IoT avec différentes fonctionnalités dans le but de combler les besoins des diverses industries, allant de l'agriculture et domaine médical, jusqu'à l'industrie militaire

\subsection{Architecture}
L'architecture d'un réseau Iot se décompose en 4 couches flexibles. Chaque couche est constituée de plusieurs technologies et standards [1]. Cet aspect modulaire permet une meilleure scalabilité de l'infrastructure et une meilleure adaptation aux besoins émergeants. Les couches de ce modèle sont comme suit:
\begin{enumerate}
  \item \emph{La couche de capteurs et d'objets intelligents:} Elle est formée d'objets connectés munis de capteurs et/ou d'actionneurs. C'est la couche la plus proche de l'environnement physique, celle-ci transforme les événements générés par ce dernier en un flux d'information à temps réel, et se charge de leur transmission. Les capteurs ont différentes spécifications, comme la mesure de la température, pression, capture de mouvement …, et sont soit connectés à des passerelles à l'aide de réseaux filaires (Ethernet) ou non (Wi-Fi, Bluetooth, RFID …), soit directement à la couche applicative. Un exemple de ce type de technologie sont les WSNs, caractérisés par leur basse consommation d'énergie et grande zone de couverture.
  \item \emph{La couche réseau et passerelles:} Cette couche garantit la transmission de la masse d'information générée par la couche précédente, tout en respectant la qualité de service exigée par les applications servies. Plusieurs infrastructures et protocoles de communication ont été mis en place dans le but d'optimiser l'acheminement et traitement d'information, comme le concept de Fog Computing qu'on détaillera par la suite.
  \item \emph{La couche de gestion:} Le rôle de cette couche est de filtrer et organiser les informations, en fournissant une couche d'abstraction à l'application. Elle s'occupe de la gestion de priorité, et l'analyse de la pertinence des données. C'est aussi à ce niveau que les politiques d'anonymisation et sécurisation de données sont implémentées.
  \item \emph{La couche d'application:} Elle est située majoritairement dans des clouds ou data-centers. Les applications couvrent des domaines différents comme l'agriculture et la gestion de villes intelligentes, et d'autres plus critiques comme le domaine de la santé ou le domaine militaire. Toutefois, depuis quelques années, les efforts de recherches visent à rapprocher ces applications de la première couche du modèle.
\end{enumerate}

\subsection{Quelque domaines d'application}
L'installation d'objets intelligents s'est démocratisée depuis quelques années. On les retrouve dans les environnements suivants:
\begin{itemize}
  \item Les systèmes de sécurité et surveillance, où des caméras et capteurs de mouvements permettent de détecter et identifier toute activité suspecte.
  \item Les maisons intelligentes: ce qui décrit la connectivité des objets dans un domicile, tous œuvrent pour fournir de meilleures conditions de vie, et ceci en assurant: la régulation de températures, l'optimisation de consommation d'énergie, la détection d'incendie et le filtrage de l'air.
  \item Les systèmes de voitures autonomes dans lesquels les véhicules utilisant la voie publique peuvent s'échanger des informations, parfois critiques, et alerter les conducteurs de tout danger imminent. Ils peuvent aussi, en se basant sur les informations de géolocalisation fournies par les autres usagers, produire des recommandations de destination tout en évitant les points de congestion de la circulation routière.
  \item Les dispositifs permettant la surveillance médicale de personnes incapacités, et ainsi prise de dispositions nécessaires dans les moments d'urgence.
\end{itemize}

\subsection{Problèmes et difficultés}
Avec la croissance du nombre d'objets connectés, les attaques cybercriminelles deviennent excentriques. Ce réseau peut être vulnérable contre l'injection de données erronées qui peuvent influencer des prises de décision parfois critiques. Les nœuds intercommunicants disposent de ressources limitées, et ainsi, ils peuvent être sujets à des attaques de déni de service DDOS. La grande quantité de données produite doit être protégée tout au long du circuit liant les couches présentées précédemment. Un travail de supervision doit être mené dans le but de garantir la confidentialité des données récoltées, et empêcher les pratiques abusives comme la vente de données d'utilisateurs.
Le provisionnement en énergie est devenu une des préoccupations de la société actuelle. On cherche à optimiser l'utilisation de ressources énergétiques, soit pour étendre l'autonomie des objets et capteurs mobiles, ou bien pour réduire les frais d'approvisionnement en électricité. Des algorithmes de gestion et allocation des ressources ont été proposés pour minimiser cette consommation, mais les recherches sur ce sujet sont toujours actives.
De plus, certaines applications demandent une certaine réactivité et une grande vitesse de réponse pour accomplir des missions critiques comme pour la conduite de véhicules automobiles ou la surveillance médicale. Ces exigences sont souvent restreintes par d'autres contraintes comme la mobilité des objets connectés ou la congestion dans le réseau de communication.