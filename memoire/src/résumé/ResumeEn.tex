\textsc{\textbf{Abstract :}}

Fog computing, also called Edge Computing, is intended for distributed computing where numerous "peripheral" devices connect to a cloud. (The word "fog" suggests a cloud's periphery or edge). Many of these devices will generate voluminous raw data (e.g., from sensors), and rather than forward all this data to cloud-based servers to be processed, the idea behind fog computing is to do as much processing as possible using computing units co-located with the data-generating devices, so that processed rather than raw data is forwarded, and bandwidth requirements are reduced. An additional benefit is that the processed data is most likely to be needed by the same devices that generated the data, so that by processing locally rather than remotely, the latency between input and response is minimized. This idea is not entirely new: in non-cloud-computing scenarios, special-purpose hardware (e.g., signal-processing chips performing Fast Fourier Transforms) has long been used to reduce latency and reduce the burden on a CPU.

\textsc{\textbf{Keywords :}} XXXX - YYYYY - ZZZZ .\\ \\