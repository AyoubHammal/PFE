\large{\textsc{\textbf{Résumé :}}}

Le fog computing, l'informatique géodistribuée, l'informatique en brouillard, ou encore l'infonébulisation, consiste à exploiter des applications et des infrastructures de traitement et de stockage de proximité, servant d'intermédiaire entre des objets connectés et une architecture informatique en nuage classique. Le but est d'optimiser les communications entre un grand nombre d'objets connectés et des services de traitement distants, en tenant compte d'une part des volumes de données considérables engendrés par ce type d'architecture (mégadonnées) et d'autre part de la variabilité de la latence dans un réseau distribué, tout en donnant un meilleur contrôle sur les données transmises.\\

\textsc{\textbf{Mots-clefs :}} XXXX - YYYYY - ZZZZ .\\ \\