\newacronym{IoT}{IoT}{signifie Internet of Things (ou Internet des objets)}
\newglossaryentry{latence}{
	name={Latence},
	description={Temps nécessaire à un paquet de données pour passer de la source à la destination à travers un réseau},
}
\newglossaryentry{Fog Computing}{
	name={Fog Computing},
	description={Technologie Cloud dans laquelle les données générées par les terminaux ne sont pas directement téléchargées dans le cloud, mais sont au préalable prétraitées dans des mini-centres de calcul décentralisés},
}
\newglossaryentry{Cloud Computing}{
	name={Cloud Computing},
	description={Correspond à l’accès à des services informatiques via Internet à partir d’un fournisseur de services},
}
\newglossaryentry{virtualisation}{
	name={Virtualisation},
	description={Technologie qui permet de créer plusieurs environnements simulés ou ressources dédiées à partir d'un seul système physique},
}
\newglossaryentry{machine virtuelle}{
	name={Machine Virtuelle},
	description={Environnement virtuel qui fonctionne comme un système informatique virtuel, avec son propre processeur, sa mémoire, son interface réseau et son espace de stockage, mais qui est créé sur un système matériel physique},
}
\newglossaryentry{conteneur}{
	name={Conteneur},
	description={Enveloppe virtuelle qui permet de distribuer une application avec tous les éléments dont elle a besoin pour fonctionner : fichiers source, environnement d'exécution, librairies, outils et fichiers},
}
\newglossaryentry{migration}{
	name={Migration},
	description={Processus qui consiste à déplacer l'état d'une instance virtuelle, d'un hôte physique à un autre}
}